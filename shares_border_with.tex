\section{Соседние страны}

Построим граф соседних стран.

\begin{itemize}
    \item Объект: \href{https://www.wikidata.org/wiki/Q6256}{страна (Q6256)},
    \item Свойство: \href{https://www.wikidata.org/wiki/Property:P47}{имеет границы с (P47)}.
\end{itemize}

\begin{lstlisting}[language=SPARQL]
#neighboring countries graph
#defaultView:Graph
SELECT ?country ?countryLabel ?sharesBorderWith ?sharesBorderWithLabel
WHERE
{
    ?country wdt:P31 wd:Q6256.                         #countries

    SERVICE wikibase:label { bd:serviceParam wikibase:language "ru" }
    OPTIONAL { ?country wdt:P47 ?sharesBorderWith . }  #shares border with
}
\end{lstlisting}

\href{https://query.wikidata.org/#%23neighboring%20countries%20graph%0A%23defaultView%3AGraph%0ASELECT%20%3Fcountry%20%3FcountryLabel%20%3FsharesBorderWith%20%3FsharesBorderWithLabel%0AWHERE%0A%7B%0A%20%20%20%20%3Fcountry%20wdt%3AP31%20wd%3AQ6256.%0A%0A%20%20%20%20SERVICE%20wikibase%3Alabel%20%7B%20bd%3AserviceParam%20wikibase%3Alanguage%20%22ru%22%20%7D%0A%20%20%20%20OPTIONAL%20%7B%20%3Fcountry%20wdt%3AP47%20%3FsharesBorderWith%20.%20%7D%0A%0A%7D%0A}{SPARQL-запрос}, 787 записей.

В результате выполнения скрипта мы получаем граф  с 787 ребрами.


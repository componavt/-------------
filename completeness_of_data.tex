\subsection{Полнота Викиданных}
Проанализируем полноту Викиданных.
\begin{itemize}
    \item По данным "Общероссийского классификатора стран мира" за 2016 год на земле существует 251 страна.~\cite{oksm}
    \item В этой задаче не учитываются древние, уже не существующие государства (например: \href{https://www.wikidata.org/wiki/Q41137}{Ассирия (Q41137)}), поскольку они являются экземпляром не объекта "country", а объекта "former country" (бывшие страны). Отметим, что количество бывших стран на порядок больше существующих ныне стран (см. \href{https://query.wikidata.org/#SELECT%20%3Fitem%20%3Flabel%20%3F_image%20WHERE%20%7B%0A%20%20%3Fitem%20wdt%3AP31%20wd%3AQ3024240.%0A%20%20SERVICE%20wikibase%3Alabel%20%7B%0A%20%20%20%20bd%3AserviceParam%20wikibase%3Alanguage%20%22en%22%20.%20%0A%20%20%20%20%3Fitem%20rdfs%3Alabel%20%3Flabel%0A%20%20%7D%0A%20%20%0AOPTIONAL%20%7B%20%3Fitem%20wdt%3AP18%20%3F_image.%20%7D%0A%7D%0A}{SPARQL-запрос}, возвращающий более двух тысяч таких стран).

    \item По данным категории \href{https://ru.wikipedia.org/wiki/%D0%90%D0%BB%D1%84%D0%B0%D0%B2%D0%B8%D1%82%D0%BD%D1%8B%D0%B9_%D1%81%D0%BF%D0%B8%D1%81%D0%BE%D0%BA_%D1%81%D1%82%D1%80%D0%B0%D0%BD_%D0%B8_%D1%82%D0%B5%D1%80%D1%80%D0%B8%D1%82%D0%BE%D1%80%D0%B8%D0%B9}{"Алфавитный список стран и территорий"} Русской Википедии существует  252 страны. (В "Общероссийском классификаторе стран мира" недостает \href{https://www.wikidata.org/wiki/Q1246}{Косово})

    \item По данным категории \href{https://www.wikipedia.org/wiki/en:List_of_sovereign_states}{"List of sovereign states"} Английской Википедии существует 206 стран.
    \end{itemize}

Не всегда можно точно указать дату основания страны по разным причинам: отсутствие, недостаток или проиворечие писменных источников. Например, основание Древнерусского государства связывают с призванием варяжского князя Рюрика в 862 году, но точной даты нет (объект \href{https://www.wikidata.org/wiki/Q159} {Russia (Q159)}). Так же некоторым современным странам предшествовал ряд других и дату образования какого из них считать за дату создания современной страны -- это вопрос открытый (например, \href{https://www.wikidata.org/wiki/Q711}{Монголия (Q711)}).


\subsection{Страны с незаполненной датой основания}

Выведем списк стран с пустым свойством "дата основания":

\begin{itemize}
    \item Объект: \href{https://www.wikidata.org/wiki/Q6256}{страна (Q6256)},
    \item Свойство: \href{https://www.wikidata.org/wiki/Property:P571}{дата основания (P571)}.
\end{itemize}

\begin{lstlisting}[language=SPARQL label=countrywithoutinception, caption=Список стран с пустым свойством "дата создания"]
#List of `instances of` "countries without an inception" 
SELECT ?country ?countryLabel 
WHERE
{
    ?country wdt:P31 wd:Q6256.       #country
    
    MINUS { ?country wdt:P571 [] } . #inception of country is empty
    SERVICE wikibase:label { bd:serviceParam wikibase:language "en" }
}
\end{lstlisting}

\href{https://query.wikidata.org/#%23List%20of%20%60instances%20of%60%20%22countries%20without%20a%20inception%22%20%0ASELECT%20%3Fcountry%20%3FcountryLabel%20%0AWHERE%0A%7B%0A%20%20%20%20%3Fcountry%20wdt%3AP31%20wd%3AQ6256.%20%23country%0A%20%20%20%20%0A%20%20%20%20MINUS%20%7B%20%3Fcountry%20wdt%3AP571%20%5B%5D%20%7D%20.%20%23inception%20of%20country%0A%20%20%20%20SERVICE%20wikibase%3Alabel%20%7B%20bd%3AserviceParam%20wikibase%3Alanguage%20%22en%22%20%7D%0A%7D%0A%0A}{SPARQL-запрос}, 100 записей.

Итак, на 6 марта 2017 года Викиданные содержат 100 из 198 записей о ныне существующих странах с неизвестным годом основания страны.
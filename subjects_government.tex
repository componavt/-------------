\section{Формы правления стран}

Построим пузырьковую диаграмму форм правления стран.

\begin{itemize}
    \item Объект: \href{https://www.wikidata.org/wiki/Q6256}{страна (Q6256)},
    \item Свойство: \href{https://www.wikidata.org/wiki/Property:P122}{форма правления (P122)}.
\end{itemize}

\begin{lstlisting}[language=SPARQL]
#basic form of government ranking
#defaultView:BubbleChart
SELECT ?bfog ?form (count(*) as ?count)
WHERE 
{
    ?country wdt:P31 wd:Q6256.  #country
    ?country wdt:P122 ?bfog .   #basic form of government
    OPTIONAL {
		?bfog rdfs:label ?form
		filter (lang(?form) = "ru")
	}
}
GROUP BY ?bfog ?form
ORDER BY DESC(?count) ASC(?form)
\end{lstlisting}

\href{}{SPARQL-запрос}, 30 записей.


Основные формы правления стран: республика (в 20 странах), конституционная монархия (в 18 странах), федеративная республика (в 18 странах), парламентская республика (в 17 странах) и	президентская республика (в 12 странах).\ref{117698}
